%%%%%%%%%%%%%%%%%%%%%%%%%%%%%%%%%%%%%%%%%
% MUW Poster
% LaTeX Template
% Version 1.0 (31/08/2016)
% (Based on Version 1.0 (31/08/2015) of the Jacobs Portrait Poster
%
% License:
% CC BY-NC-SA 3.0 (http://creativecommons.org/licenses/by-nc-sa/3.0/)
%
% Created by:
% Nicolas Ballarini, CeMSIIS, Medical University of Vienna
% nicoballarini@gmail.com
% http://statistics.msi.meduniwien.ac.at/
%%%%%%%%%%%%%%%%%%%%%%%%%%%%%%%%%%%%%%%%%


\def\footer#1{\def\insertfooter{#1}}
%--------------------------------------------------------------------------------------
%	PACKAGES AND OTHER DOCUMENT CONFIGURATIONS
%--------------------------------------------------------------------------------------

\documentclass[final]{beamer}



\usepackage[scale=1.150]{beamerposter} % Use the beamerposter package
\usetheme{MUWposter} % Use the MUWposter theme supplied with this template

% Include a logo of your project if desired
\logo{\pgfputat{\pgfxy(-15,108)}{\pgfbox[center,base]{\includegraphics[width=12cm]{pbn.png}}}}  


\usepackage{multicol}
\usepackage{array}
%The following two are column definitions for the aknowledgements section
\newcolumntype{L}{>{\arraybackslash}m{22cm}}
\newcolumntype{S}{>{\arraybackslash}m{5cm}}
\usepackage{pgf}  
\usepackage{mathtools}
\usepackage{amsmath, amsthm, amssymb, amsfonts}
\usepackage{exscale}
\usepackage{xcolor}
\usepackage{ushort}
\usepackage{setspace}
\usepackage[square,numbers]{natbib}
\usepackage{url}
\bibliographystyle{abbrvnat}
\renewcommand{\vec}[1]{\ushort{#1}}
\renewcommand{\vec}[1]{\mathbf{#1}}
\definecolor{greenMUW}{RGB}{60,191,174}
\definecolor{blueMUW}{RGB}{17,29,79}
\definecolor{skinMUW}{RGB}{254,228,217}
\definecolor{hellblauMUW}{RGB}{95,180,229}

%-----------------------------------------------
%  START Set the colors
%  Uncomment to apply colors you want to use.
%-----------------------------------------------
\colorlet{themecolor}{greenMUW}
\usebackgroundtemplate{}

%\colorlet{themecolor}{skinMUW}
%\colorlet{themecolor}{blueMUW}
%\usebackgroundtemplate{\includegraphics{MUW_skin.pdf}}

%%\colorlet{themecolor}{blueMUW}
%\colorlet{themecolor}{hellblauMUW}
%\usebackgroundtemplate{\includegraphics{MUW_hellblau.pdf}}
%-----------------------------------------------
%  END Set the colors
%-----------------------------------------------


%-----------------------------------------------
%  START Set the width of the columns
%-----------------------------------------------
\setlength{\paperwidth}{33.1in} % A0 width: 46.8in
\setlength{\paperheight}{46.8in} % A0 height: 33.1in
\newlength{\sepmargin}
\newlength{\sepwid}
\newlength{\onecolwid}
\newlength{\twocolwid}
\newlength{\threecolwid}

% The following measures are used for 2 columns
\setlength{\sepmargin}{0.055\paperwidth} % Separation width (white space) between columns
\setlength{\sepwid}{0.03\paperwidth} % Separation width (white space) between columns
\setlength{\onecolwid}{0.43\paperwidth} % Width of one column
\setlength{\twocolwid}{0.9\paperwidth} % Width of two columns

%-----------------------------------------------------------
% The following measures are used for 3 columns
%\setlength{\sepmargin}{0.06\paperwidth} % Separation width (white space) between columns
%\setlength{\sepwid}{0.02\paperwidth} % Separation width (white space) between columns
%\setlength{\onecolwid}{0.28\paperwidth} % Width of one column
%\setlength{\twocolwid}{0.58\paperwidth} % Width of two columns
%\setlength{\threecolwid}{0.88\paperwidth} % Width of three columns
%\setlength{\columnsep}{30pt}

%-----------------------------------------------
%  END Set the width of the columns
%-----------------------------------------------

\usepackage[absolute]{textpos}
%--------------------------------------------------------------------------------------
%	TITLE SECTION 
%--------------------------------------------------------------------------------------
\setbeamertemplate{title}[left]
\setbeamertemplate{frametitle}[default][left]
%\setmainfont{Georgia}

\title{Creating Knowledge graphs from the literature: the case of health resilience in Green Building Neighbourhoods} % Poster title

\author{Luc Jonveaux, Filip Kučera, Jorge Velasco Manrique, Marta Ingelmo Gomez, Kathrine Nykjær Brejnrod,  Sissa Bekombo Priso} % Author(s)

\institute{The Integrator-centric approach for realising innovative energy efficient buildings in connected sustainable green neighbourhoods project} % Institution(s)
%--------------------------------------------------------------------------------------



\begin{document}
\textblockorigin{10mm}{10mm} % start everything near the top-left corner
%% https://cordis.europa.eu/project/id/101037075/
  \addtobeamertemplate{block end}{}{\vspace*{1ex}} % White space under blocks
  \addtobeamertemplate{block alerted end}{}{\vspace*{0ex}} % White space under highlighted (alert) blocks
  \setlength{\belowcaptionskip}{2ex} % White space under figures
  \setlength\belowdisplayshortskip{1ex} % White space under equations
  
  


      \begin{columns}[t] % The whole poster consists of two major columns
	  
      \begin{column}{\sepmargin}\end{column}
      
	    \begin{column}{\onecolwid} % The first column


		  \begin{block}{Context}
          %\begin{multicols}{2}
The Horizon 2020 \href{https://doi.org/10.3030/101037075}{PROBONO project} (Grant agreement  101037075) aims at demonstrating "\textit{strong examples of how Green Building Neighbourhoods (GBNs) technological and social innovations can be applied, with a vision focused on building infrastructure and a renewed focus on people and sustainability, taking full advantage of digitization and smart technologies for the benefit of society}". The Task 3.5 of this projects  aims at reviewing "Interventions to mitigate diseases outbreaks".

In this document, we summarize how we used new technologies, including Large Language Models (LLMs), to consolidate a Knowledge Graph (KG) of this topic, based on the literature, to demonstrate the feasibility of building a body of knowledge pertaining to a certain domain, with a specific angle. This opens the door to more opportunities the growing space existing between LLMs and KGs.
\end{block}

\begin{block}{Disclaimer}
Please note that this document is a research-based exploration compiled by knowledge management researchers, not medical professionals. Our findings are presented with the intention to inform and contribute to the dialogue on public health strategies. They are indicative and should serve as a preliminary guide. We encourage all readers to consult with qualified health professionals for expert advice and to confirm and enrich these insights.

          %\end{multicols}
          \end{block}

          
          
          \begin{block}{Results}
          %\begin{multicols}{2}

We started with defining a basic ontology based on classes of interest for mitigation measures, namely 'Risks', 'Mitigations', 'People', and 'Technologies'. These constitute an initial body of knowledge, which is then used to build 'Blueprints', possible interventions to mitigate diseases outbreaks. 

To date, the Knowledge Graph (v0.4) contains information on 377 articles, from which were programmaticaly derived 21145  risks, 22950  mitigation measures, 16125  stakeholders and 23140  technologies. The team used these to build 24 blueprints manually, and automated the production of 50 others.

We hope that releasing the knowledge graph under an open-source license  (\href{https://creativecommons.org/licenses/by-nc-sa/4.0/deed.en}{CC BY-NC-SA}) will drive use of this knowledge graph and that health professionals can use this to derive useful, professionally-approved mitigation measures.

          %\end{multicols}
          \end{block}
         
         \end{column}
                  
                  
        % SEparator column
         \begin{column}{\sepwid}  \end{column}
         
         
         
         
         \begin{column}{\onecolwid} %The second column
                   
          \begin{block}{Objective}
The main objective of the task is to review the scientific literature to identify key risks, stakeholders, technologies and mitigations measures both at building and neighbourhood scales.

          \end{block}
         \begin{block}{Technical details}
          %\begin{multicols}{2}
          The present knowledge graph has been created using new tools, helping to streamline, faster and more consistently, information from the literature:
           \begin{itemize}
            \item Parsing of the literature was done with \textit{GROBID}. This provided structured text (XML) from the article PDFs;
            \item The data was processed and structured in an RDF that was produced with the \textit{Owlready2} python library;
            \item Vector embedding based on \textit{ChromaDB} because of the early possibility to integrate with \textit{LangChain}, and because of its ease of use.
            
 \end{itemize}
Once the data was prepared, we explored structuring the information using different solutions:
           \begin{itemize}
\item \textit{NLTK} was used to extract topics and themes of the articles and \textit{Spacy} with and \textit{CoreferenceResolver} to tackle disambiguation ;
\item Text was processed using a combination of OpenAI API (both using GPT3.5 and GPT4 endpoints) as well as running local models (NOUS/LLaMa), using the python requests or \textit{LangChain} libraries.
\item We used the content from the articles, stored in \textit{ChromaDB}, used through \textit{LangChain} and deployed through a \textit{FastAPI API}.
          %\end{multicols}
                     \end{itemize}

                               \end{block}
          
          
          \begin{block}{Next steps}
          
                     This activity yielded expected outcomes, consisting in mapping out risks, technologies and stakeholders, as well as suggesting mitigation measures. This however is an asset that has a highly reusable potential, and this list, albeit listing possible actions from a project perspective, could be undertaken by other parties:
                                \begin{itemize}
\item We plan on integrating more robust graph management solutions, possibly Neo4j or similar, to continually review and enrich the KG;
\item We would want to enrich the semantic content of the graph to make it more usable and possible an input to KG-backed LLMs;
\item Reusing existing semantic assets (eg Wikidata) might help structure
\item Connect this knowledge graph to other KGs or ontologies.
     %\end{multicols}
                     \end{itemize}
          \end{block}
          
      \end{column}
      
      \begin{column}{\sepmargin} \end{column}
      \end{columns} 
       
      \begin{columns}[t] % Split up the two columns wide column again
      
      \begin{column}{\sepmargin} \end{column}
        \begin{column}{\onecolwid} % The first column
			\begin{block}{\large Acknowledgements}
                    Thank you to the PROBONO team.
                    %\begin{center}
				%		\begin{tabular}{SL}
			%				\includegraphics[width=\linewidth]{Flag_of_Europe.png}  &
			%				\footnotesize This project has received funding from the  grant agreement No 11111.
			%			\end{tabular}
			%		\end{center}
				\end{block}	
                \vspace*{-0.9cm}
				\begin{alertblock}{\large Contact Information}
                \vspace*{-0.5cm}
					\begin{footnotesize}
					\begin{itemize}
						\item \href{mailto:luc.jonveaux@mottmac.com}{contact@probonoh2020.eu}
						\item \href{https://www.probonoh2020.eu/}{www.probonoh2020.eu}
					\end{itemize}
					\end{footnotesize}	
					
				\end{alertblock}
		    \end{column} % End of the first column
			\begin{column}{\sepwid}\end{column} % Empty spacer column
			\begin{column}{\onecolwid} % Begin a column 
              \begin{block}{\large Solution repository}
			  %\vspace*{-0.5cm}
              	\nocite{*} % Insert publications even if they are not cited in the poster
					
                    	%\bibliographystyle{plainurl}
						\href{https://github.com/mm80843/T3.5}{GitHub repository}
      
				\end{block} 
    \begin{block}{\large Note(s)}
			  %\vspace*{-0.5cm}
              	\nocite{*} % Insert publications even if they are not cited in the poster
					%{\footnotesize
                    	%\bibliographystyle{plainurl}
						This PDF can be renamed as a ZIP file to extract source code for this document and for the solution.
      % @LJ: For this, see https://github.com/ansemjo/truepolyglot
      %}
				\end{block} 
			\end{column} % End of the second column
            
			\begin{column}{\sepmargin}\end{column} % Empty spacer column
            
            
\end{columns} % End of all the columns in the poster

\begin{picture} (-5cm,-5cm) (-3.5cm, 14.5cm)
\put(0.5cm, 0.5cm) {\includegraphics [width=5cm]{eu.png}} 
\end{picture}
\begin{textblock}{12}(1.8,15.24)
{\fontsize{28pt}{24pt}\selectfont  This project has received funding from the European Union’s Horizon 2020 Europe Research and Innovation programme under Grant Agreement No 101037075. This output reflects only the author’s view, and the European Union cannot be held responsible for any use that may be made of the information contained therein.}
\end{textblock}

\end{document}

